\section{命题时序逻辑}
\subsection{语法}
\noindent\textcircled{\footnotesize{1}}~notation: \\
   \indent Z:整数;\\
   \indent N:正整数;\\
   \indent $N_0$:非负整数;\\
   \indent Prop:可数原子命题;\\
   \indent operate:${^\lnot},\square,\bigcirc,\Diamond,;,\vee,\wedge,+,*$

\noindent\textcircled{\footnotesize{2}}~归纳定义命题公式:
\begin{equation}\nonumber
\begin{split}
&1.~~p\in Prop,p ~is ~a ~formula.\\
&2.~~p,q \in Prop,so ~are ~the ~constructs.\\
&~~~~~~~~~~p,q,p\wedge q,p;q,\bigcirc p,\square q,p^+,q^*.and~so~on.
\end{split}
\end{equation}

\subsection{语义}
\noindent\textcircled{\footnotesize{1}}~状态~S~,~B=\{true,false\}\\
\indent ~S~:~Prop~$\rightarrow$~B~~~~~~~~~~\{原子命题~$\rightarrow$~B\}\\
\indent ~状态是原子命题到真假集合的映射。

\noindent\textcircled{\footnotesize{2}}~区间\\
\indent 区间~$\sigma$~:~非空状态序列。\\
\indent 区间长度~:~~~有限区间~~~~~~finite~~~~$|\sigma|$~~,~~$|\sigma|=status-1$ \\
\indent~~~~~~~~~~~~~~~~~~~~~无限区间~~~~infinite~~~~w\\
\indent~~~~~~~~~~~~~~~~~~~~~统一{\color{red}{$\bigstar$}},~$N_w=N_0\vee\{w\}$

operator~~$=,<,\leq,\preceq$\\
\indent~~~~对于~~~~$N_w~~,~w=w$\\
\indent~~~~对于~$i\in N_0,~i<w$\\
\indent~~~~$\preceq~~as~~\leq~-\{{w,w}\}$\\
\indent定义:\\
\indent~~~~1.~$\sigma~as~<s_0,...,s_{|\sigma|}>$~,~若~$\sigma~is~infinite.~~s_{|\sigma|}$~无定义。\\
\indent~~~~2.~$\sigma_{(i,...j)}~as~(0\leq i\preceq ~\leq |\sigma|)~<s_i,...s_j>$\\
\indent~~~~3.~$\sigma^{(k)}~as~(0\leq k \preceq |\sigma|)~<s_k,...,s_{|\sigma|}>$\\


%\subsection{可满足性与有效性}
%\subsection{时序操作符}
%\subsection{操作符的优先规则}
%\subsection{等价关系}
%\subsection{逻辑法则}




%\newpage
%\section{一阶时序逻辑}
%\subsection{语法}
%\subsection{语义}
%\subsection{可满足性与有效性}
%\subsection{逻辑法则}