\section{PTL练习题}

\noindent常用基本符号优先级:\\
$1.~~{^\lnot}$\\
$2.~\bigcirc,\Diamond,\square,+,*$\\
$3.~~\wedge,\vee$\\
$4.~\rightarrow,\leftrightarrow$\\
$5.~~;$

\textbf{2018/7/22}

\noindent题~1~:~存在原子命题A,B,将下列命题用PTL公式表示:\\
\textcircled{\footnotesize{1}}~~A发生,那么之后B总会发生(包括当前状态)。\\
\textcircled{\footnotesize{2}}~~A发生,那么之前B总会发生(包括当前状态)。\\
答:\\
\textcircled{\footnotesize{1}}~~$\square(A\rightarrow\Diamond B)$ \\
{\color{red}{\textcircled{\footnotesize{2}}~~$^\lnot(\square^\lnot B;A)$}\\
理解:chop 分为两个区间,两个区间上分别真假共四种情况,而此时否定一种,剩三种,与\textcircled{\footnotesize{1}}的蕴含效果一样,具体情况为$ ^\lnot(P;Q) $被这样的区间满足:对于该区间的任意一种两段划分,要么第一段不满足P,要么第二段不满足Q (可以 第一段不满足P 同时 第二段不满足Q)。}

\textbf{2018/7/23}

\noindent题~2~:~存在原子命题A,B,假设有个词语"严格之后"表示之后的状态,但不包括当前状态。
"严格之前"同理,将下列命题用PTL公式表示:\\
\textcircled{\footnotesize{1}}~~A发生,那么严格之后B总会发生(不包括当前状态)。\\
\textcircled{\footnotesize{2}}~~A发生,那么严格之前B总会发生(不包括当前状态)。\\
答:\\
\textcircled{\footnotesize{1}}~~$\square(A\rightarrow\bigcirc\Diamond B)$ \\
\textcircled{\footnotesize{2}}~~$^\lnot((\square(more\wedge{^\lnot B}));A)$~{\color{red}{$\times$}~~~~$~\square~$和~more连用表示无穷区间~}\\
\textcircled{\footnotesize{2}}~~$^\lnot((\square(\circleddash{^\lnot B}));A)$~{\color{red}{$\times$}~~~~$\circleddash$不能和always直接连用,因为第一个状态没有上一状态。}\\
\textcircled{\footnotesize{2}}~~$({^\lnot}A)\wedge((\Diamond B;\bigcirc A)\vee(\square{^\lnot A}))$~{\color{red}{?}}\\
\textcircled{\footnotesize{2}}~~$({^\lnot}A)\wedge{^\lnot(\square{^\lnot B;skip;A})}$~${\color{red}{\surd}}$~~~~{\color{red}{题目的隐含意思有A不可能发生在起始状态}}\\
\textcircled{\footnotesize{2}}~~$({^\lnot}A)\wedge{^\lnot(\square{^\lnot B;\bigcirc A})}$~${\color{red}{\surd}}$~~~~{\color{red}{题目的隐含意思有A不可能发生在起始状态}}\\

\newpage
\textbf{2018/7/25 ~Kun~整理笔记}
\begin{center}
\textbf{时序逻辑性质表达 }~~~~~-2018/5/21
\end{center}
是朋友才能发消息,如何表示?\\
是朋友:~f~;发消息:~s~。\\

\noindent\textcircled{\footnotesize{1}}~~经典逻辑表达:\\
a).~~~~$s\rightarrow f$~~~~~发消息必是朋友。\\
b).~${^\lnot} f\rightarrow {^\lnot}s$~~~不是朋友不能发消息\\
c).~${^\lnot} ( {^\lnot} f\wedge s )$~~~不是朋友且发消息是不对的~~(与b)等价)\\
d).~~~~$f\vee {^\lnot s}$~~~~~要么是朋友要么不发消息~~(与b),c)等价)\\



\noindent\textcircled{\footnotesize{2}}~~时序逻辑表示:\\
a).将来能发消息,现在一定是朋友。
$$\Diamond(\Diamond s \rightarrow f) ~weak~~~~~~~~~~\square(s\rightarrow f)$$
b).现在不是朋友,将来不能发消息。
$$\square({^\lnot} f \rightarrow{^\lnot} s)$$
c).现在不是朋友,将来能发消息是不对的。
$${^\lnot}\Diamond(\square{^\lnot}f;s)~~~~~~~~~~\square{^\lnot({^\lnot} f\wedge s)}$$
d).任何时刻,要么是朋友,要么不发消息。
$$\square(f\vee{^\lnot}s)$$
e).在是朋友之前不能发消息。
$$\Diamond(s\rightarrow\boxdot f)$$


\noindent Runtime~verification~for~LTL~and~TLTL ~:\\
$${^\lnot}s~until~f\equiv \square {^\lnot}s;skip;f$$




%\[
%  \text{实数} \begin{cases}
%    \text{有理数}{\begin{cases}
%      \text{整数}{\begin{cases}
%          \text{奇数} \\ \text{偶数}
%        \end{cases}}\\
%      \text{分数}
%    \end{cases}} \\[4ex]
%    \text{无理数}{\begin{cases}
%      \text{代数无理数} \\ \text{超越数}
%    \end{cases}}
%  \end{cases}
%\]
%
%\[
%  \text{实数} \begin{cases}
%    \text{有理数}\smash[t]{\begin{cases}
%      \text{整数}\smash{\begin{cases}
%          \text{奇数} \\ \text{偶数}
%        \end{cases}}\\
%      \text{分数}
%    \end{cases}} \\[4ex]
%    \text{无理数}\smash[b]{\begin{cases}
%      \text{代数无理数} \\ \text{超越数}
%    \end{cases}}
%  \end{cases}
%\]
%
%\[
%    \text{机器学习} \begin{cases}
%        \text{~~监督学习~}{\begin{cases}
%            \text{回归算法} \\
%            \text{分类算法}{\begin{cases}
%            \text{生成模型} \\
%            \text{判别模型}
%        \end{cases}}
%        \end{cases}} \\
%        \text{半监督学习} \\
%        \text{非监督学习}{\begin{cases}
%            \text{关联规则学习} \\
%            \text{~~~聚类算法~~~}
%        \end{cases}}  \\
%        \end{cases}
%\]
%
%\[
%    \text{机器学习} \begin{cases}
%        \text{~~监督学习~}\smash[t]{\begin{cases}
%            \text{回归算法} \\
%            \text{分类算法}\smash{\begin{cases}
%            \text{生成模型} \\
%            \text{判别模型}
%        \end{cases}}
%        \end{cases}} \\
%        \text{半监督学习} \\
%        \text{非监督学习}\smash[b]{\begin{cases}
%            \text{关联规则学习} \\
%            \text{~~~聚类算法~~~}
%        \end{cases}}  \\
%        \end{cases}
%\]
